
\usepackage{etoolbox}
\usepackage{ifthen}
\usepackage{xifthen}




\newcommand{\ifnotempty}[2]{\ifthenelse{\isempty{#1}}{}{#2}}

% \def\xcoloropts{usenames,dvipsnames}
% \DeclareOption{noxcoloropts}{\def\xcoloropts{}}
% \DeclareOption*{\PackageWarning{jhupllab}{Unknown option: \CurrentOption}}
% \ProcessOptions\relax
% \usepackage[dvipsnames,xcdraw]{xcolor}

%%%%%%%%%%%%%%%%%%%%%%%%%%%%%%%%%%%%%%%%%%%%%%%%%%%%%%%%%%%%%%%%%%%%%%%%%%%%%%%
%%% Document mode routines

\makeatletter
\def\documentmode@{draft}
\newcommand{\documentModeFinal}{\def\documentmode@{final}}
\newcommand{\documentModeDraft}{\def\documentmode@{draft}}
\newcommand{\isdraft}{\equal{\documentmode@}{draft}}
\newcommand{\isfinal}{\equal{\documentmode@}{final}}
\newcommand{\draftonly}[1]{\ifthenelse{\isdraft}{#1}{}}
\makeatother

%%%%%%%%%%%%%%%%%%%%%%%%%%%%%%%%%%%%%%%%%%%%%%%%%%%%%%%%%%%%%%%%%%%%%%%%%%%%%%%
%%% Rule for mathpartir

\usepackage{mathpartir}

\def\MathparLineskip{\lineskip=.25cm} % limits space between rules, rule names already spacing them
\newcommand{\bbrule}[4][]{\inferrule*[left={\bbrulename{#2}},#1]{#3}{#4}}
\newcommand{\bbrulename}[1]{{\sc #1}}

\newcommand{\fnstyle}[1]{\text{\textsmaller{\sc{#1}}}}
\newcommand{\deffn}[2]{%
    \expandafter\newcommand\expandafter{\csname #1\endcsname}[1][]{\fnstyle{#2}\ifthenelse{\isempty{##1}}{}{(##1)}}%
}


%%%%%%%%%%%%%%%%%%%%%%%%%%%%%%%%%%%%%%%%%%%%%%%%%%%%%%%%%%%%%%%%%%%%%%%%%%%%%%%
%%% Inline notes

% Defines a textual style for inline notes.
\newcommand{\notestyle}[1]{\textnormal{\scriptsize \itshape #1}}

% Determines whether notes are enabled for the document.
\newif\ifnotes
\notestrue % notes default to enabled
\newcommand{\nonotes}{\notesfalse} % disables notes

% Defines a command to create new note commands.  This command can be run once
% for each contributor to define the command prefix, contributor initials, and
% color of the note text.

\makeatletter
\newcommand{\defnote}[3]{%
    \expandafter\newcommand\expandafter{\csname #1note\endcsname}[1]{%
        % See http://tex.stackexchange.com/questions/294693/multiple-space-hacks-generate-more-space
        % for an explanation of this spooky whitespace-handling code.
        \@bsphack
        %=== instead of \@esphack: ===
        \relax
        \ifhmode
            \spacefactor\@savsf
            \ifdim\@savsk>\z@
            \nobreak
            \hskip\z@skip
            % The previous action will change \lastskip, so:
            \hskip-\@savsk
            \hskip\@savsk      
            % now \lastskip is almost \@savsk again.
            \ignorespaces
            \fi
        \fi
        %===========================
        \ifnotes%
            \draftonly{\notestyle{\color{#3}(##1 -- #2)}}%
        \fi
    }%
}
\makeatother

% Define note commands for the members of the JHU PL lab.
\definecolor{goldenrod}{rgb}{0.85, 0.65, 0.13}
\definecolor{rawsienna}{rgb}{0.839, 0.541, 0.349}
\defnote{s}{SS}{goldenrod} % Scott Smith
\defnote{z}{ZP}{ForestGreen} % Zachary Palmer
\defnote{w}{SW}{rawsienna} % Shiwei Weng
\defnote{h}{HM}{rawsienna} % Hari Menon
\defnote{l}{LF}{VioletRed} % Leandro Facchinetti

%%%%%%%%%%%%%%%%%%%%%%%%%%%%%%%%%%%%%%%%%%%%%%%%%%%%%%%%%%%%%%%%%%%%%%%%%%%%%%%
%%% Utilities for language grammars

% Defines an environment for displaying language grammars.
% Within this environment, some commands are defined:
%   \grule[description]{non-terminal}{productions}
%   \gor (a symbol for separating productions)
%   \gline (a symbol for breaking a line of productions if more space is needed)
%   \setGrammarVertAdjustment{length} (which sets the vertical adjustment for
%                                      the next grammar newline)
\newcommand{\grammarNoteSpace}{\hspace{4mm}}
\newcounter{grammarnote}
\setcounter{grammarnote}{0}
\newcommand{\grammarDefs}{%
    \global\def\grammarVertAdjustment{0mm}%
    \newcommand{\gcomment}[1]{\hfill%
        \ifnum\value{grammarnote}=0
            \stepcounter{grammarnote}
            \grammarNoteSpace \textrm{\textsmaller{\itshape ##1}}
        \fi
    }%
    \newcommand{\grule}[3][]{%
        \setcounter{grammarnote}{0}
        ##2 & \ifnotempty{##3}{::=} & \newcommand{\gcommenttext}{##1} ##3 \hfill \gcomment{##1} \endgrule
    }%
    \newcommand{\gskip}{&&\endgrule}%
    \def\endgrule{\\[\grammarVertAdjustment]}%
    \newcommand{\gor}{\mathbin{\vert}}%
    \newcommand{\gline}{%
        \hfill \gcomment{\gcommenttext} \\[\grammarVertAdjustment] &&
    }%
}
\def\grammarColPad{\quad}%
\newenvironment{grammar}{%
    \begingroup%
    \grammarDefs%
    \begin{math}\begin{array}{@{}r@{\grammarColPad}c@{\grammarColPad}l@{}}%
}{%
    \end{array}\end{math}%
    \endgroup%
}
\newcounter{grammarrulecount}%
\newenvironment{grammarTwoCol}{%
    \begingroup%
    \grammarDefs%
    \setcounter{grammarrulecount}{0}%
    \renewcommand{\endgrule}{%
        \stepcounter{grammarrulecount}%
        \ifnum\value{grammarrulecount}=2%
            \setcounter{grammarrulecount}{0}%
            \\[\grammarVertAdjustment]%
        \else%
            &%
        \fi%
    }%
    \begin{cmarray}{@{}r c l @{\qquad} r c l@{}}%
}{
    \end{cmarray}%
    \endgroup%
}

% Defines a command for creating grammar terminal macros.  Grammar terminal macros always display their contents in true-type font.
% The second argument defines the text of the command while the first (optional) argument defines the command's name.  If the name is
% not specified, it is equal to the text as processed by the \defgtname command.  By default, this command will produce its input
% prefixed with a lower-case "gt" (for "grammar terminal"); for instance, \defgt{x} defines a command \gtx which prints an x.
\newcommand{\defgt}[2][]{%
    \ifthenelse{\isempty{#2}}{%
        \warn{Ignoring empty argument to \char`\\defgt}%
    }{%
        \ifthenelse{\isempty{#1}}{\defgt[\defgtname{#2}]{#2}}{%
            \expandafter\newcommand\expandafter{\csname #1\endcsname}{\mathinner{\texttt{\gttext{#2}}}}
        }%
    }%
}
\newcommand{\defgtname}[1]{gt#1}
\newcommand{\gttext}[1]{#1}

% Defines a command for creating grammar non-terminal macros.  Grammar non-terminal macros display their contents in math font.
% As above, the command name is derived by default from the text; in this case, the command is \defgnname, the default implementation
% for which prefixes the text with "gn".
\newcommand{\defgn}[2][]{%
    \ifthenelse{\isempty{#2}}{%
        \warn{Ignoring empty argument to \char`\\defgn}%
    }{%
        \ifthenelse{\isempty{#1}}{\defgn[\defgnname{#2}]{#2}}{%
            \expandafter\newcommand\expandafter{\csname #1\endcsname}{\ensuremath{#2}}
        }%
    }%
}
\newcommand{\defgnname}[1]{gn#1}

%
% macros
%

\defgn[gre]{e}
\defgn[lab]{\ell}
\defgn[grv]{v}
\defgn[grx]{x}

\newcommand{\stk}{\Sigma}
\newcommand{\envt}{\sigma}
\newcommand{\steps}{\Rightarrow}

\newcommand{\la}{\langle}
\newcommand{\ra}{\rangle}

\defgt[gtarrow]{\texttt{\upshape ->}}
\defgt[gtfun]{\upshape fun}

\newcommand{\gfun}{\gtfun \grx \gtarrow \gre}
\newcommand{\gclo}{\la\gfun, \sigma\ra}


% \defgt[gtclo]{\langle \gtfun \sigma \rangle}
% \defgt[gtclo]{\gtfun \envt}
